\documentclass[12pt,letterpaper]{article}
\usepackage{preamble}

\newcommand\course{MS108}
\newcommand\userID{518030910440}
\setcounter{section}{0}
\begin{document}
%\textbf{\Large RISC-V Report}
\section{Abstract}
    This is a report of the project of MS108 homework implementing a simplified CPU supporting some of the RISC-V instruction sets. 
    My solution is based on Tomasulo algorithm. 
\section{Feature}
\subsection{Tomasulo}
    The Tomasulo algorithm aims to execute Out-of-Order(abbreviated OoO), 
    and solve problems triggered by OoO, such as WAR, WAW, and so on. The key is renaming, 
    RS and ROB. 

    In my design, each FU has its own RS, and the size of each RS is different. 
    Renaming is based on where the instruction is in ROB. 
\subsection{Cache}
    I implemented a 512B I-cache in my project, then I made it a 2-way associative one, but there is little improvement. 
    I guess this is because the testbench only contains short codes. 

    I've also implemented a D-cache with writing back. This performs well in simulation, 
    but costs a great amount of logic units(20\% more LUT) so I give up. 

    My 2-way associative cache and D-cache are in /backup/Units. 
\subsection{Branch Policy}
    I first designed one without branch prediction and its policy is stall. Since branch instructions hold about sixth of instructions, 
    this one stalls frequently and is not so OoO. 

    Then I made one with branch prediction. Like what BOOM(Berkeley Out-of-Order Machine) does, it uses shadow registers and branch masks. 
    Each instruction has a branch mask to evidence if it follows in-flight branch instructions. After executing a branch instruction, 
    it boardcasts to update others' branch masks. Only Execute load/store instructions with cleared branch mask. Branch prediction policy is simply GShare. 
    
    This one can run with a frequency of 100MHz. (I slow down to 80MHz with no neg delay)

    This sounds cool, but when I implemented it, I found the delay high, since to deal with the boardcast of branch instructions, 
    a large number of logic units are needed. (let alone the shadow registers cost considerable sources too)

    But it runs faster and is really COOOOOOOOL. 
\section{Summary}
    1. The main problem I face is not understanding how to make improvement in organization and hardware level. 
    Once I noticed that I had repeated calculating one same thing in each RS, so I picked it out: 
    calculate it outside and send the result to each RS. 

    The result was crazy: the number of LUT increased considerably. 

    2. The Tomasulo is significantly efficient when handling a set of load/store instructions. 

    3. The biggest drawbck is unsupporting precise exceptions, as it does not handle branch misprediction in ROB. 
    Other can-be improvements include superscalar and superpipeline. 

    4.1 Pitfall: sometimes bigger and dumber is better. (CAAQA, ed.6, charpter three)

    4.2 Pitfall: and sometimes smarter is better than bigger and dumber. (from ditto)
\section{Reference}
1. \url{https://docs.boom-core.org/en/latest/}, an out of order RISC-V CPU using chisel, with branch bits(named branch mask there)

2. \url{http://www.kroening.com/diplom/diplom/main003.html}, details about the hardware of Tomasulo Architecture.
\section{Design}
\begin{figure}[H]
    \centering
    \includegraphics[width=165mm]{design.jpg}
\end{figure}
\end{document}